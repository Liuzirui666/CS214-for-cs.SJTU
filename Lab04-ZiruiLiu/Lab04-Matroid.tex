\documentclass[12pt,a4paper]{article}
\usepackage{amsmath,amscd,amsbsy,amssymb,latexsym,url,bm,amsthm}
\usepackage{epsfig,graphicx,subfigure}
\usepackage{enumitem,balance}
\usepackage{wrapfig}
\usepackage{mathrsfs,euscript}
\usepackage[usenames]{xcolor}
\usepackage{hyperref}
\usepackage[vlined,ruled,linesnumbered]{algorithm2e}
\usepackage{float}
%\usepacksge{amssymb}
\usepackage{amsmath}
\hypersetup{colorlinks=true,linkcolor=black}

\newtheorem{theorem}{Theorem}
\newtheorem{lemma}[theorem]{Lemma}
\newtheorem{proposition}[theorem]{Proposition}
\newtheorem{corollary}[theorem]{Corollary}
\newtheorem{exercise}{Exercise}
\newtheorem*{solution}{Solution}
\newtheorem{definition}{Definition}
\theoremstyle{definition}

\renewcommand{\thefootnote}{\fnsymbol{footnote}}

\newcommand{\postscript}[2]
 {\setlength{\epsfxsize}{#2\hsize}
  \centerline{\epsfbox{#1}}}

\renewcommand{\baselinestretch}{1.0}

\setlength{\oddsidemargin}{-0.365in}
\setlength{\evensidemargin}{-0.365in}
\setlength{\topmargin}{-0.3in}
\setlength{\headheight}{0in}
\setlength{\headsep}{0in}
\setlength{\textheight}{10.1in}
\setlength{\textwidth}{7in}
\makeatletter \renewenvironment{proof}[1][Proof] {\par\pushQED{\qed}\normalfont\topsep6\p@\@plus6\p@\relax\trivlist\item[\hskip\labelsep\bfseries#1\@addpunct{.}]\ignorespaces}{\popQED\endtrivlist\@endpefalse} \makeatother
\makeatletter
\renewenvironment{solution}[1][Solution] {\par\pushQED{\qed}\normalfont\topsep6\p@\@plus6\p@\relax\trivlist\item[\hskip\labelsep\bfseries#1\@addpunct{.}]\ignorespaces}{\popQED\endtrivlist\@endpefalse} \makeatother

\begin{document}

\noindent

%========================================================================
\noindent\framebox[\linewidth]{\shortstack[c]{
\Large{\textbf{Lab04-Matroid}}\vspace{1mm}\\
CS214-Algorithm and Complexity, Xiaofeng Gao, Spring 2021.}}
\begin{center}
\footnotesize{\color{red}$*$ If there is any problem, please contact TA Haolin Zhou.}

% Please write down your name, student id and email.
\footnotesize{\color{blue}$*$ Name:Zirui Liu  \quad Student ID:519021910343 \quad Email: L.prime@sjtu.edu.cn}
\end{center}

\begin{enumerate}
\item \textit{Property of Matroid.} 
\begin{enumerate}
	\item
	Consider an arbitrary undirected graph $ G=(V,E) $. Let us define $ M_{G}=(S,C) $ where $ S=E $ and $ C=\left\{I \subseteq E \mid\left(V, E \backslash I\right) \text { is connected}\right\} $. Prove that $ M_{G} $ is a \textbf{matroid}.\par
	    \begin{proof}
	        %Uncomment this block to write your solution.
	        To prove that $M_G$ is a matroid, it has to satisfy those three conditions. For the first,$S=E$, and obviously $S$ is not empty. For the second, since $\left(V, E \backslash I\right) \text {is connected} $, so $\left(V, E\right)$ must be connected too. Since adding more edges to a graph that is already connected would not make it disconnected, for any $B \in C$, and $A \subseteq B$, we have $A \in C$. For the third, for any $A \in C, B \in C, |A| < |B|$, similar to the way we prove the second condition, since $C$ can be defined as " the power set featuring all the edges as elements, for $X \in B-A $, we can always have $A \cup \left\{ X \right\} \in C$. So in conclusion, $ M_{G} $ is a \textbf{matroid}. 
	        
	    \end{proof}
	\item
	Given a set $A$ containing $n$ real numbers, and you are allowed to choose $k$ numbers from $A$. The bigger the sum of the chosen numbers is, the better. What is your algorithm to choose? Prove its correctness using \textbf{matroid}.\par
	\textbf{Remark:} Denote $\mathbf{C}$ be the collection of all subsets of $A$ that contains no more than $k$ elements. Try to prove $(A,\mathbf{C})$ is a matroid.\par
	    \begin{solution}
	        %Uncomment this block to write your solution.
	        The solution is as following: We sorted the $n$ numbers, then we choose from them from the biggest to the smallest until we have chosen $k$ numbers. These $k$ numbers are those we wish to find. Next I will prove the correctness of the greedy algorithm. 
	        
	        We suppose that $M = \left(U, F\right)$ is a weighted matrioid and that $U$ is sorted into monotonically decreasing order. We also assume that $ \omega \left( x \right)$ represents the weight of element $x$. Then we let $x$ to be the biggest element of $U$ such that $\left\{ x \right\} \in F $. If any such $x$ exists, then there exists an optimal subset $A$ of $U$ containing $x$, thus the optimizism of the greedy algorithm can be proved. The proof is as following. 
	        
	        If no such $x$ exists, then the only independent subset is the empty set $ \text {like} \left( F = \left\{ \varnothing \right\}\right)$ then the question is meaningless. We assume that $F$ contains another non-empty optimal subset
            $B$, then there exists two possibilities: $x \in B $ or $x \notin B$. In the first case, we can simply make $A = B$ to prove the correctness. In the second case, we 
            assume that $x \notin B$.
            Then we assume there exists $y \in B$ so that $\omega\left(y\right) > \omega\left(x\right)$. Since $y \in B$ and $B \in F$ so we can get $\left\{y\right\} \in F$.
            If $\omega\left(y\right) > \omega\left(x\right)$, then $y$ would be the first element of $U$, making a contradiction with our original assumption.

            So, $ \forall y \in B, \omega\left(x\right) \ge \omega\left(y\right)$.
            We can now construct a set $A \in F$ such that $x \in A,|A| =|B|$, and $\omega\left(A\right) \ge \omega\left(B\right)$. We then use the second characteristic of \textbf {Matroid}, we can always find an element of $B$ to add to $A$ without destructing the independence.
            We can repeat this process until we get $|A| =|B|$. Then by construction we have $A = B - \left\{y\right\} \cup \left\{x\right\} $ for $y \in B$. Then since $\omega\left(A\right) \ge \omega\left(B\right)$,
            we have $\omega\left(A\right) = \omega\left(B\right) - \omega\left(y\right) + \omega\left(x\right) \ge \omega\left(B\right)$. 
            Since $B$ is optimal, then $A$ containing $x$ is also optimal. So the correctness of our algorithm is proved.

	        
	
	    \end{solution}
\newpage

\end{enumerate}
\item \textit{Unit-time Task Scheduling Problem.} Consider the instance of the \textbf{Unit-time Task Scheduling Problem} given in class. 
    \begin{enumerate}
        \item Each penalty $\omega_{i}$ is replaced by $80-\omega_{i}$. The modified instance is given in Tab.~\ref{tab:1}. Give the final schedule and the optimal penalty of the new instance using Greedy-MAX.
		\begin{table}[H]
			\setlength{\abovecaptionskip}{0.cm}
			\setlength{\belowcaptionskip}{0.5cm}
			\centering
			\caption{Task}
			\label{tab:1}			
			\begin{tabular}{|c|ccccccc|}
				\hline
				$ a_{i} $&1&2&3&4&5&6&7\\
				\hline
				$ d_{i} $&4&2&4&3&1&4&6\\
                \hline
                $ \omega_{i} $&10&20&30&40&50&60&70\\
				\hline
			\end{tabular}
		\end{table}
	        \begin{solution}
	            %Uncomment this block to write your solution.
	            The optimal penalty is 30 for the following schedule: $  a_{5}\Rightarrow a_{6}\Rightarrow a_{4}\Rightarrow a_{3}\Rightarrow a_{7}\Rightarrow a_{2}\Rightarrow a_{1}   $
	        \end{solution}
        \item Show how to determine in time $O(|A|)$ whether or not a given set $A$ of tasks is independent. (\textbf{Hint}: You can use the lemma of equivalence given in class)
 	        \begin{solution}
                %Uncomment this block to write your solution.
                To determine in time $O(|A|)$ whether or not a given set $A$ of tasks is independent, we can use the lemma of equivalence. That is to say, we need to prove that the tasks in $A$ are scheduled in order of monotonicallly increasing deadlines. So what we have to do is to check if all the tasks in $A$ whether their deadline time is in monotonicallly increasing order or not.
                On the other hand, we can solver the problem by using the lemma: set $A$ is independent if and only if for any $t=0,1,2,...n$, $N_t\left(A\right) \leq t$. The pseudo code is as follows:
        \begin{algorithm}[H]
		\KwIn{An array $a[1,\cdots,n]$}
		\KwOut{a bool value, indicating $A$ is independent or not.}
		
		\BlankLine
		\caption{}\label{Algorithm}
		
		
% 		sort(x,x+n)\;
%         int start=x[0]\;
%         int end=start+k\;
%         int index=0,num=0\;
%         \While{index<=n-1}{
%             \While{x[index]<=end}{
%                 index++\;        
%             }
%             num++\;
%             start=x[index]\;
%             end=start+k\;
%         }
%         return num\;
        
        \For{$i \leftarrow 1$ \KwTo $n$}{
        $a\left[i\right] \leftarrow a\left[i\right]+1$\;
        }
        $Nt$ \leftarrow $0$\;
        
        \For{$i \leftarrow 1$ \KwTo $n$}{
        $N \leftarrow Nt+a\left[i\right]$\;
            \If{$Nt>i$}{
            return $false$\;
            }
        }
        return $true$;
        
            		
% 		$pivot \leftarrow A[n]$; $i \leftarrow 1$\;
% 		\For{$j \leftarrow 1$ \KwTo $n-1$}{
% 			\If{$A[j] < pivot$}{
% 				swap $A[i]$ and $A[j]$\;
% 				$i \leftarrow i+1$\;
% 			}
% 		}
		
% 		swap $A[i]$ and $A[n]$\;
% 		\lIf{$i>1$}{$\operatorname{QuickSort}(A[1,\cdots,i-1])$}
% 		\lIf{$i<n$}{$\operatorname{QuickSort}(A[i+1,\cdots,n])$}
		
		
	    \end{algorithm}

            \end{solution}
    \end{enumerate}

\item \textit{MAX-3DM.} Let $X$, $Y$, $Z$ be three sets. We say two triples $\left(x_{1}, y_{1}, z_{1}\right)$ and $\left(x_{2}, y_{2}, z_{2}\right)$ in $X \times Y \times Z$ are \textit{disjoint} if $x_{1} \neq x_{2}$, $y_{1} \neq y_{2},$ and $z_{1} \neq z_{2}$. Consider the following problem:
    
    \begin{definition}[MAX-3DM] 
        Given three disjoint sets $X$, $Y$, $Z$ and a non-negative weight function $c(\cdot)$ on all triples in $X \times Y \times Z$, \textbf{Maximum 3-Dimensional Matching} (MAX-3DM) is to find a collection $\mathcal{F}$ of disjoint triples with maximum total weight.
    \end{definition}

    \begin{enumerate}
    	\item Let $D = X \times Y \times Z$. Define independent sets for MAX-3DM.
    	\item Write a greedy algorithm based on Greedy-MAX in the form of \emph{pseudo code}. \label{Item-Greedy}
    	\item Give a counter-example to show that your Greedy-MAX algorithm in Q.~\ref{Item-Greedy} is not optimal.
    	\item Show that: $\max\limits_{F \subseteq D} \frac{v(F)}{u(F)} \leq 3$. {\color{blue}(Hint: you may need Theorem~\ref{Thm-Intersect} for this subquestion.)} 
    	    \begin{solution}
    	        %Uncomment this block to write your proof.
    	        \\(1):\\
    	        We define that: $\forall \left(x_i,y_i,z_i\right) and \left(x_j,y_j,z_j\right) \in \mathcal{F}, \left(x_i,y_i,z_i\right) and \left(x_j,y_j,z_j\right) $ are disjoint. Then we have this $\mathcal{F}$ to be independent.
    	        \\(2):\\
    	We define $\left(x_i,y_i,z_i\right)$ as $a_i$\\
    	\begin{algorithm}[H]
		\KwIn{An array $a[1,\cdots,n]$}
		\KwOut{An array $A$, indicating the result we want.}
		
		\BlankLine
		\caption{}\label{Algorithm}
        sort $\mathcal{F}$ by $c$ in non-increasing order\;
        $A \leftarrow \varnothing$\;
        \For{$i \leftarrow 1$ \KwTo $n$}{
            \If{$A \cup \left\{a_i\right\} \in \mathcal{F}$}{
             $A \leftarrow A \cup \left\{d_i\right\}$\;
            }
        }
        return $A$
        
            		
% 		$pivot \leftarrow A[n]$; $i \leftarrow 1$\;
% 		\For{$j \leftarrow 1$ \KwTo $n-1$}{
% 			\If{$A[j] < pivot$}{
% 				swap $A[i]$ and $A[j]$\;
% 				$i \leftarrow i+1$\;
% 			}
% 		}
		
% 		swap $A[i]$ and $A[n]$\;
% 		\lIf{$i>1$}{$\operatorname{QuickSort}(A[1,\cdots,i-1])$}
% 		\lIf{$i<n$}{$\operatorname{QuickSort}(A[i+1,\cdots,n])$}
		
		
	    \end{algorithm}
    	        
    	        \\(3):\\
    	        We give that: 
    	        $a_1 = \left(x_1,y_1,z_1\right) = \left(a,b,c\right)$, having the weight of $10$.
    	        $a_2 = \left(x_2,y_2,z_2\right) = \left(a,e,f\right)$, having the weight of $9$.
    	        $a_3 = \left(x_3,y_3,z_3\right) = \left(g,b,k\right)$, having the weight of $8$.
    	        for any other $a_i$, their total weight conbined is less than $5$. By the greedy algorithm we should choose $a_1$ since it has the largest weight, but the best solution is choosing $a_1$, (whether we consider other elements does not matter), but the best choice is choosing $a_2$ and $a_3$.
    	        
    	        \\(4):\\ %   \mathcal{I}_{1}
    	        According to $Theorem 1$, we can transform the question to proving that $(E, \mathcal{I}_{i})$ for $i=1,2,3$ are  \textbf{matroids}. We start by constructing three sets for $\mathcal{I}_{i}$.\\
    	        $\mathcal{I}_{1}$: $\forall \left(x_i,y_i,z_i\right) and \left(x_j,y_j,z_j\right) \in E$, $x_i \neq x_j$.\\
    	       $\mathcal{I}_{2}$: $\forall \left(x_i,y_i,z_i\right) and \left(x_j,y_j,z_j\right) \in E$, $y_i \neq y_j$.   \\  	            	        $\mathcal{I}_{3}$: $\forall \left(x_i,y_i,z_i\right) and \left(x_j,y_j,z_j\right) \in E$, $z_i \neq z_j$.\\
    	       And $\textbf{\mathcal{I}}=\bigcap_{i=1}^{k} \mathcal{I}_{i}$.\\
    	       $\textbf{Heredity:}$\\
    	       It is easy to understand that the heredity can always be satisfied.\\
    	       $\textbf{Exchange Property:}$\\
    	       For $A,B \subseteq \mathcal{I}_{1}$, $|A| < |B|$, there exist $a_i = \left(x_i,y_i,z_i\right) \in B-A$, considering that $\mathcal{I}_{1}$: $\forall \left(x_i,y_i,z_i\right) and \left(x_j,y_j,z_j\right) \in E$, $x_i \neq x_j$, so $ \forall \left(x_j,y_j,z_j\right) \in A$, $x_i \neq x_j$, so $A\cup \left\{a_i\right\} \in \mathcal{I}_{1}$. Thus the exchange property for
    	       $\left(E, \mathcal{I}_{1}\right)$ is proved. The same as $\left(E, \mathcal{I}_{2}\right)$ and $\left(E, \mathcal{I}_{3}\right)$ by using $y_i$ and $z_i$. So in conclusion, $(E, \mathcal{I}_{i})$ for $i=1,2,3$ are  \textbf{matroids}, so $\max\limits_{F \subseteq D} \frac{v(F)}{u(F)} \leq 3$. 
    	        
    	    \end{solution}
    \end{enumerate}
    \begin{theorem} \label{Thm-Intersect}
        Suppose an independent system $(E, \mathcal{I})$ is the intersection of $k$ matroids $\left(E, \mathcal{I}_{i}\right)$, $1 \leq i \leq k$; that is, $\mathcal{I}=\bigcap_{i=1}^{k} \mathcal{I}_{i}$. Then $\max\limits_{F \subseteq E} \frac{v(F)}{u(F)} \leq k$, where $v(F)$ is the maximum size of independent subset in $F$ and $u(F)$ is the minimum size of maximal independent subset in $F$.
    \end{theorem}    
\end{enumerate}

\vspace{20pt}

\textbf{Remark:} You need to include your .pdf and .tex files in your uploaded .rar or .zip file.

%========================================================================
\end{document}
