\documentclass[12pt,a4paper]{article}
\usepackage{ctex}
\usepackage{amsmath,amscd,amsbsy,amssymb,latexsym,url,bm,amsthm}
\usepackage{epsfig,graphicx,subfigure}
\usepackage{enumitem,balance}
\usepackage{wrapfig}
\usepackage{mathrsfs,euscript}
\usepackage[usenames]{xcolor}
\usepackage{hyperref}
\usepackage[vlined,ruled,linesnumbered]{algorithm2e}
\usepackage{array}
\hypersetup{colorlinks=true,linkcolor=black}

\newtheorem{theorem}{Theorem}
\newtheorem{lemma}[theorem]{Lemma}
\newtheorem{proposition}[theorem]{Proposition}
\newtheorem{corollary}[theorem]{Corollary}
\newtheorem{exercise}{Exercise}
\newtheorem*{solution}{Solution}
\newtheorem{definition}{Definition}
\theoremstyle{definition}

\renewcommand{\thefootnote}{\fnsymbol{footnote}}

\newcommand{\postscript}[2]
 {\setlength{\epsfxsize}{#2\hsize}
  \centerline{\epsfbox{#1}}}

\renewcommand{\baselinestretch}{1.0}

\setlength{\oddsidemargin}{-0.365in}
\setlength{\evensidemargin}{-0.365in}
\setlength{\topmargin}{-0.3in}
\setlength{\headheight}{0in}
\setlength{\headsep}{0in}
\setlength{\textheight}{10.1in}
\setlength{\textwidth}{7in}
\makeatletter \renewenvironment{proof}[1][Proof] {\par\pushQED{\qed}\normalfont\topsep6\p@\@plus6\p@\relax\trivlist\item[\hskip\labelsep\bfseries#1\@addpunct{.}]\ignorespaces}{\popQED\endtrivlist\@endpefalse} \makeatother
\makeatletter
\renewenvironment{solution}[1][Solution] {\par\pushQED{\qed}\normalfont\topsep6\p@\@plus6\p@\relax\trivlist\item[\hskip\labelsep\bfseries#1\@addpunct{.}]\ignorespaces}{\popQED\endtrivlist\@endpefalse} \makeatother

\begin{document}
\noindent

%========================================================================
\noindent\framebox[\linewidth]{\shortstack[c]{
\Large{\textbf{Lab11-NP Reduction}}\vspace{1mm}\\
CS214-Algorithm and Complexity, Xiaofeng Gao \& Lei Wang, Spring 2021.}}
\begin{center}
\footnotesize{\color{red}$*$ If there is any problem, please contact TA Yihao Xie. }

\footnotesize{\color{blue}$*$ Name:Zirui Liu  \quad Student ID:519021910343 \quad Email: L.prime@sjtu.edu.cn}
\end{center}

\begin{enumerate}
    \item We are feeling experimental and want to create a new dish. There are various ingredients we can choose from and we'd like to use as many of them as possible, but some ingredients don't go well with others. If there are $n$ possible ingredients (numbered 1 to $n$), we write down an $n\cdot n$ matrix giving the discord between any pair of ingredients. This discord is a real number between 0.0 and 1.0, where 0.0 means "they go together perfectly" and 1.0 means "they really don't go together." Here's an example matrix when there are five possible ingredients.
    \begin{center}
        \begin{tabular}{|c|ccccc|}
        \hline
             & 1  & 2 & 3 & 4 & 5\\
        \hline
            1 & 0.0 & 0.4 & 0.2 & 0.9 & 1.0\\
            2 & 0.4 & 0.0 & 0.1 & 1.0 & 0.2\\
            3 & 0.2 & 0.1 & 0.0 & 0.8 & 0.5\\
            4 & 0.9 & 1.0 & 0.8 & 0.0 & 0.2\\
            5 & 1.0 & 0.2 & 0.5 & 0.2 & 0.0\\
        \hline
        \end{tabular}
    \end{center}
    In this case, ingredients 2 and 3 go together pretty well whereas 1 and 5 clash badly. Notice that this matrix is necessarily symmetric; and that the diagonal entries are always 0.0. Any set of ingredients incurs a penalty which is the sum of all discord values between pairs of ingredients. For instance, the set of ingredients $(1,3,5)$ incurs a penalty of $0.2+1.0+0.5 = 1.7$. We define the \textsc{Experimental Cuisine} as follows:

        Given $n$ ingredients to choose from, the $n\times n$ discord matrix and integer $k$ and a number $p$,  decide whether there exists a collection of at least $k$ ingredients that has a penalty $\leqslant p$

    Prove that $\textsc{3-SAT}\leq_p\textsc{Experimental Cuisine}$
    
    \begin{solution}
    We will show that 3-SAT problem can be changed into Experimental Cuisine problem. Assume that for any 3-SAT problem which made of $n$ clauses, % \vee
    for one of the clauses that $A=\left(x \vee y \vee z \right)$, we then have $7$ ingredients, representing $7$ different cases for $A$ to be satisfied. But these $7$ cases can not be correct at the same time, so it is safe for us to set all the discord between these $7$ ingredients to be $1$. For different clauses of ingredients, if they have conflicts, discord is set to be $1$. If NO CONFLICTS, DISCORD WILL BE SET $0$. When we can choose enough ingredients, whose number of clauses equal to $n$, then this 3-SAT problem can be satisfied. Thus 3-SAT can be transformed into Experimental Cuisine problem. When solving the EXPERIMENTAL CUISINE problem, for an ingredient, if it is added to the selection and the size of p does not exceed the set value, then it can be selected, otherwise it cannot. Each judgment costs polynomial time, and the result can be obtained after a polynomial number of judgments, then the whole process can be completed in polynomial time. Therefore,in conclusion, if EXPERIMENTAL CUISINE is solvable in polynomial time, so is 3-SAT.

    
    \end{solution}
    
    
    \item An induced subgraph $G'=(V',E')$ of a graph $G=(V,E)$ is a graph that satisfies $V'\subseteq V$ and $E' =\{(u,v)\in E| u,v\in V'\}$. Given two graphs $G_1=(V_1,E_1)$ and $G_2=(V_2,E_2)$ and an integer $b$, we need to decide whether $G_1$ and $G_2$ have a common induced subgraph $G_c$ with at least $b$ nodes. This problem is called \textsc{Maximum Common Subgraph} (MCS). Prove that MCS is NP-complete. (Hint: reduce from \textsc{INDEPENDENT-SET})

    \begin{solution}
    First note that the maximum common subgraph problem is in the class of NP. To verify a given solution $V_1^'$ and $V_2^'$ along with a proposed isomorphism mapping $V_1 \setminus V_1^'$ to $V_2\setminus V_2^'$, we will simply need to delete these sets of vertices and their incident edges from the graphs $G_1$ and $G_2$ and check
    whether the proposed isomorphism is indeed bijective, which can be done in $O\left(n^2\right)$ time. This is an example
    of a problem with a less trivial step to prove that it is in NP, and you will not be penalized if your solution
    is left somewhat informal.
    To prove NP-hardness, we reduce from the clique problem. In the clique problem, we are given a graph
    $G =\left (V, E\right)$ and a goal $g$, and are asked to find a clique of size $g$ in $G$. A clique of size $g$ is a set $g$ vertices
    such that there is an edge between every pair of vertices in the set. Given an instance of the clique problem,
    we reduce it to an instance of the MCSP as follows: create a complete graph $G_0$ which has the same vertices as
    $G$ but has an edge between all pairs of vertices. The two input graphs to our maximum common subgraph
    problem are $G$ and $G_0$,
    Suppose that $V_1^{'} \subseteq G$ and $V_2^{'} \subseteq G_0$
    is a solution to the MCSP on $G$ and $G_0$
    . If $|V | − |V_1^{'}| \geq g $, then there
    is a clique of size $g$ in $G$. Since $G_0$
    is a complete graph, any subgraph must also be a complete graph,and
    thus a clique. So if there is a common subgraph of $G$ and $G_0$ of size at least $g$, then this must be a clique of
    size at least $g$. In particular, $V \backslash V_1^'$
    is a clique of size at least $g$ and one can trivially throw away vertices to
    get a clique of size exactly $g$. Otherwise, there is no clique of size $g$ in $G$. On the other hand, we suppose that there is
    a clique $V_0$ of size $g$ in $G$. Then clearly it forms a common subgraph of size $g$ with any subgraph of $G_0$ of size
    $g$. The reduction is in polynomial time because it just requires creating a complete graph with a polynomial
    number of edges and vertices with respect to $G$. So in conclusion,  MCS is NP-complete.

    
    
    \end{solution}

    
    \item Let us define the $k$-spanning tree as a spanning tree in which each node has a degree $\leqslant k$. Given a graph $G= (V,E)$ and a positive integer $k$, we need to decide whether there exists a $k$-spanning tree in $G$. Prove that this problem is NP-complete. (Hint: reduce from \textsc{HAMILTONIAN-CYCLE})
    
    \begin{solution}
    First, we note that 2-SPANNING-TREE is exactly undirected 
    HAMILTONIAN-PATH: a tree in which each vertex has degrees at most $2$ is a path. Thus, 2-SPANNING-TREE is NP-hard.
Now, we give a reduction from 2-SPANNING-TREE to k-SPANNING-TREE for any $k \geq 3$, which exactly means reduction from HAMILTONIAN-CYCLE. Given
a graph $G\left(V, E\right)$ as an instance of 2-SPANNING-TREE, we construct another graph $G^{'}\left(V^{'},E^{'}\right)$
where $V^{'}$
contains $V$ and some other vertices as follows. For each vertex $v \in V $, $V^{'}$ also contains
$k − 2$ new vertices $v_1, v_2, . . . v_{k−2}$. Similarly,  $E^{'}$
contains $E$ and some extra edges connecting $v_i$
and $v$. Formally,\\
\begin{equation}
    V^{'} = V \bigcup \left\{ v_1, v_2, . . . v_{k−2} | v \in V  \right\}  \\
\end{equation}

\begin{equation}
    E^{'} = E \bigcup \left\{ e_1, e_2, . . . e_{k−2} | e \in E  \right\}
\end{equation}
    We will claim that $G$ has a 2-SPANNING-TREE if and only if $G_0$ has a k-SPANNING-TREE. On one hand, we assume that $G$ has a 2-SPANNING-TREE $T$, then    $ T^{'} = T \bigcup \left\{ v_iv | v \in V , i=1,2 ... k-2 \right\} $ is a
spanning tree of $G^{'}$
. Since each vertex of $T$ has degree at most $2$, each vertex of $T^{'}$ has degree at
most $k$.
On the other hand, assume that $G^{'}$ has a k-SPANNING-TREE $T^{'}$
. One can see that $T^{'}$ must
contains all the edges $v_iv$, since those are the only edges incident at $v_i$
. Furthermore, all $v_i$
’s are leaves of $T^{'}$
since they all have degree $1$. Thus we can remove them to obtain a spanning tree $T$ of
$G$. After removing all edges $v_iv$ decrease the degree of each $v$ by $k−2$, $T$ is a 2-SPANNING-TREE
of $G$.
In conclusion, k-SPANNING-TREE is NP-complete for any $k \geq 2$.

    
    
    \end{solution}
    
    \item We define the decision problem of \textsc{Knapsack Problem} as follows:
    
        Given $n$ indivisible objects, each with a weight of $w_i>0$ kilograms and a value $v_i>0$, a knapsack with capacity of $W$ kilograms and a number $k$, decide whether there is a collection of objects that can be put into the knapsack with a total value $V\geqslant k$.
        
    Prove that \textsc{Knapsack Problem} is NP-complete.
    
    \begin{solution}
    %p = \sum_{n=1}^Na_n
    First of all, we will proof that Knapsack is NP. The proof is the set $S$ of items that are chosen and the
    verification process is to compute $\sum_{i \in S} s_i $ and $\sum_{i \in S} v_i $
    , which takes polynomial time for the size of
    input.\\
Secondly, we will show that there is a polynomial reduction from Partition problem to Knapsack, thus leading to our problem, determining Knapsack to be NPC.
It suffices to show that there exists a polynomial time of reduction $Q\left(·\right)$ such that $Q\left(X\right)$ is a ‘Yes’
instance to Knapsack if and only if $X$ is a ‘Yes’ instance to Partition. Suppose we are given $a_1, a_2, . . . , a_n$ for
the Partition problem, consider the following Knapsack problem: $s_i = a_i$, $v_i = a_i$ for $i = 1, . . . , n$,
$B = V = \frac{1}{2} \sum_{i=1}^n a_i       $. $Q(·)$ here is the process converting the Partition problem to Knapsack
problem. It is clear that this process is polynomial in the input size.
If $X$ is a ‘Yes’ instance for the Partition problem, there exists $S$ and $T$ such that $\sum_{i \in S} a_i = \sum_{i \in T} a_i = \frac{1}{2} \sum_{i=1}^n a_i$. We then let our Knapsack contain the items in $S$, and it follows that  $\sum_{i \in S} s_i = \sum_{i \in S} a_i = B$ and  $\sum_{i \in S} v_i = \sum_{i \in S} a_i = V$. Therefore, $Q(X)$ is a ‘Yes’ instance for the Knapsack problem.\\
Conversely, if $Q(X)$ is a ‘Yes’ instance for the Knapsack problem, with the chosen set $S$, let $T=\left\{1,2, ...n\right\}$. We have $\sum_{i \in S} s_i = \sum_{i \in S} a_i \leq B = \frac{1}{2} \sum_{i=1}^n a_i$ and $\sum_{i \in S} v_i = \sum_{i \in S} a_i \leq V = \frac{1}{2} \sum_{i=1}^n a_i$. This means that $\sum_{i \in S} a_i = \frac{1}{2} \sum_{i=1}^n a_i$ and $\sum_{i \in T} a_i = \frac{1}{2} \sum_{i=1}^n a_i$
Therefore, $\left\{S, T\right\}$ is the desired partition, and $X$ is a ‘Yes’ instance for the Partition problem. In conclusion, This can proof that Knapsack problem is NPC.


    
    
    \end{solution}
    
    
\end{enumerate}


\textbf{Remark:} Please include your .pdf, .tex files for uploading with standard file names.
\newpage


%========================================================================
\end{document}